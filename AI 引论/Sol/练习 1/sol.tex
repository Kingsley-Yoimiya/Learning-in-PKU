\documentclass[UTF-8]{ctexart}
\usepackage{graphicx}
\usepackage{subfigure}
\usepackage{xcolor}
\usepackage{amsmath}
\usepackage{amssymb}
\usepackage{tabularx}
\usepackage{amssymb}
\usepackage{amsthm}
\title{AI 引论作业 1}
\author{2300012929 尹锦润}
\begin{document}
\maketitle

一

记事件 $\displaystyle A_{i}$ 为选中第 $\displaystyle i$ 级选手,事件 $\displaystyle B_{i}$ 为第 $\displaystyle i$ 记选手击中十环,事件 $\displaystyle C$ 为问题描述事件。

那么


\begin{equation*}
\begin{aligned}
P( C) & =\sum P( A_{i}) P( B_{i})\\
 & =\frac{4*0.9+7*0.7+6*0.5+3*0.3}{20}\\
 & =0.62
\end{aligned}
\end{equation*}


二



记事件 $\displaystyle A_{1}$ 为选出的为男性,$\displaystyle A_{2}$ 为选出的为女性,$\displaystyle B$ 为选出的为色盲。


\begin{gather*}
P( B|A_{1}) =0.05\\
P( B|A_{2}) =0.005\\
P( A_{1}) =P( A_{2}) =0.5\\
P( B) =P( B|A_{1}) P( A_{1}) +P( B|A_{2}) P( A_{2}) =0.055*0.5=0.0275\\
P( A_{1} |B) =\frac{P( A_{1} B)}{P( B)} =\frac{0.05*0.5}{0.0275} =\frac{10}{11}
\end{gather*}


三

(1)

记 $\displaystyle A$ 为该同学第一次作业合格,$\displaystyle B$ 为该同学第二次作业合格,$\displaystyle C$ 为他可以参加期末考试。

有


\begin{gather*}
P( A) =p\\
P( B|A) =p\\
P( B|\overline{A}) =\frac{p}{4}\\
P(\overline{C}) =P(\overline{A}\overline{B}) =P(\overline{B} |\overline{A}) P(\overline{A}) =\left( 1-\frac{p}{4}\right)( 1-p)\\
P( C) =1-P(\overline{C}) =1-\left( 1-\frac{p}{4}\right)( 1-p)
\end{gather*}
(2)


\begin{gather*}
P( B) =P( B|A) P( A) +P( B|\overline{A}) P(\overline{A}) =p^{2} +\frac{p}{4}( 1-p) =\frac{3}{4} p^{2} +\frac{p}{4}\\
P( A|B) =\frac{P( AB)}{P( B)} =\frac{p^{2}}{\frac{3}{4} p^{2} +\frac{p}{4}} =\frac{4p}{3p+1}
\end{gather*}


四



分布律:
\begin{gather*}
P\{X=4\} =\frac{\binom{4}{4}}{\binom{6}{4}} =\frac{1}{15}\\
P\{X=5\} =\frac{\binom{5}{4} -\binom{4}{4}}{\binom{6}{4}} =\frac{4}{15}\\
P\{X=6\} =\frac{\binom{6}{4} -\binom{5}{4}}{\binom{6}{4}} =\frac{10}{15}
\end{gather*}

分布函数:


\begin{equation*}
F( x) =\begin{cases}
0 & ,x< 4\\
\frac{1}{15} & ,4\leqslant x< 5\\
\frac{1}{3} & ,5\leqslant x< 6\\
1 & ,6\leqslant x
\end{cases}
\end{equation*}


五




\begin{gather*}
	E(X)=3\times\frac{5}{15}=1
\end{gather*}


六



(1)


\begin{equation*}
1=2\int _{0}^{+\infty } f( x)\mathrm{d} x=2A\left(\frac{1}{-2} e^{-2x}\middle| _{0}^{+\infty }\right) =A
\end{equation*}

因此 $\displaystyle A=1$。



(2)

当 $\displaystyle x >0$ 时,$\displaystyle F( x) =\frac{1}{2} +\int _{0}^{x} f( t)\mathrm{d} t=1-\frac{e^{-2x}}{2}$。

当 $\displaystyle x\leqslant 0$ 时,$\displaystyle F( x) =\frac{1}{2} -\int _{x}^{0} f( t)\mathrm{d} t=\frac{e^{-2|x|}}{2}$。

(3)


\begin{gather*}
F( 3) =1-\frac{e^{-6}}{2}\\
F( -2) =\frac{e^{-4}}{2}\\
F( 3) -F( -2) =1-\frac{1}{2}\left( e^{-6} +e^{-4}\right)
\end{gather*}


七

(1)


\begin{equation*}
E( X) =\int _{0}^{+\infty } f( x) x\mathrm{d} x=\int _{0}^{+\infty }\left( xe^{-x}\right)\mathrm{d} x=\left. -( 1+x) e^{-x}\middle| _{0}^{+\infty }\right. =1
\end{equation*}


(2)


\begin{gather*}
E\left( X^{2}\right) =\int _{0}^{+\infty } f( x) x^{2}\mathrm{d} x=\int _{0}^{+\infty } e^{-x} x^{2}\mathrm{d} x=\left. -\left( x^{2} +2x+2\right) e^{-x}\middle| _{0}^{+\infty } \ \right. =2\\
D( X) =E\left( X^{2}\right) -E( X) =1
\end{gather*}

\end{document}
